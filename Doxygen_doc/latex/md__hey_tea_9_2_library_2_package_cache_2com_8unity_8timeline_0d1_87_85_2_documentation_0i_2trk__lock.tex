\chapter{Locking tracks}
\hypertarget{md__hey_tea_9_2_library_2_package_cache_2com_8unity_8timeline_0d1_87_85_2_documentation_0i_2trk__lock}{}\label{md__hey_tea_9_2_library_2_package_cache_2com_8unity_8timeline_0d1_87_85_2_documentation_0i_2trk__lock}\index{Locking tracks@{Locking tracks}}
\label{md__hey_tea_9_2_library_2_package_cache_2com_8unity_8timeline_0d1_87_85_2_documentation_0i_2trk__lock_autotoc_md4760}%
\Hypertarget{md__hey_tea_9_2_library_2_package_cache_2com_8unity_8timeline_0d1_87_85_2_documentation_0i_2trk__lock_autotoc_md4760}%
 Lock a track to prevent editing of the track and any of the clips used by the track.

Use lock when you have finished animating a track and you want to avoid inadvertently modifying the track. You cannot edit or delete a locked track, or select its clips. The Lock icon identifies a locked track.



{\itshape Selected and locked track with Lock icon (red circle)}

To lock a track, right-\/click on the track and select {\bfseries{Lock}} from the context menu. You can also select a track and press L. You can select and lock multiple tracks at a time. A track can be both locked and muted.

To unlock a track, click the Lock icon. You can also select a locked track and press L, or right-\/click and select {\bfseries{Unlock}}. 