\chapter{README}
\hypertarget{md__hey_tea_9_2_library_2_package_cache_2com_8unity_8inputsystem_0d1_86_83_2_samples_0i_2_input_recorder_2_r_e_a_d_m_e}{}\label{md__hey_tea_9_2_library_2_package_cache_2com_8unity_8inputsystem_0d1_86_83_2_samples_0i_2_input_recorder_2_r_e_a_d_m_e}\index{README@{README}}
This sample is both a demonstration for how to use \href{https://docs.unity3d.com/Packages/com.unity.inputsystem@latest/index.html?subfolder=/api/UnityEngine.InputSystem.LowLevel.InputEventTrace.html}{\texttt{ {\ttfamily Input\+Event\+Trace}}} as well as a useful tool by itself in the form of the \href{./InputRecorder.cs}{\texttt{ {\ttfamily Input\+Recorder}}} reusable {\ttfamily Mono\+Behaviour} component.

One possible way in which you can use this facility, for example, is to record input, save it to disk, and then replay the same input in automation (e.\+g. in tests or when recording short video snippets of preset gameplay sequences for manual proofing). 